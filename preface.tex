\chapter{Preface}

Geometric Algebra is a mathematical theory with a long history. One can track its root back to 1840's, with the early works of Grassmann and Hamilton. Their ideas were ahead of time, and it was only 40 years after that W. Clifford put together this ideas in what we now call Clifford Algebras, but his early death and the vector system proposed by Gibbs made this remain as abstract ideas for pure mathematicians. In 1960's D. Hestenes used this ideas with a geometrical interpretation to explain Space-Time Lorentz transformation, whence the name Geometric Algebra. 

\hfill \newline

Time has passed and we are 40 years since then, and a question remains open: Is Geometric Algebra a unified language for physics? I don't want to give an answer, because there are plenty of books and articles that have partially done that. The problem I had with them is that, with a few exceptions, they try to convince you that geometric algebra is a better tool than complex numbers, matrix algebra, tensors, differential forms, etc. and forget to actually do physics and give a hands on calculation.

\hfill \newline


I am in the process of writing this book, and my plan for this is the following: Write a mathematical introduction to the subject. From axioms to some really nice results like the isomorphism with the most important structures in mathematics. One feature of Geometric Algebra that seems very attractive is its coordinate-free formulations, but it is useful to know how to actually compute something when a basis is chosen.  Once this is done, then I will continue to the physics.

\hfill \newline

In the physics section is it not only about showing the nice things you can do with geometric algebra, like reducing the Maxwell's equation to a single one, but deriving the most general consequences of this equation and, again, using some numbers to actually get something more than just expressions. Sometimes a graph is all that is needed to make sense of symbols, and sometimes relate a real situation is better. It's the ``equations out of equations'' chain that makes hard to follow the reasoning, and is that is exactly what I want to fill with this book. So here we go, and as I write the book I will come here to explain what has been done and what is missing.